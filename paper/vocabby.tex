%
% File acl2018.tex
%
%% Based on the style files for ACL-2017, with some changes, which were, in turn,
%% Based on the style files for ACL-2015, with some improvements
%%  taken from the NAACL-2016 style
%% Based on the style files for ACL-2014, which were, in turn,
%% based on ACL-2013, ACL-2012, ACL-2011, ACL-2010, ACL-IJCNLP-2009,
%% EACL-2009, IJCNLP-2008...
%% Based on the style files for EACL 2006 by 
%%e.agirre@ehu.es or Sergi.Balari@uab.es
%% and that of ACL 08 by Joakim Nivre and Noah Smith

\documentclass[11pt,a4paper]{article}
\usepackage[hyperref]{acl2018}
\usepackage{times}
\usepackage{latexsym}

\usepackage{url}

%\aclfinalcopy % Uncomment this line for the final submission
%\def\aclpaperid{***} %  Enter the acl Paper ID here

%\setlength\titlebox{5cm}
% You can expand the titlebox if you need extra space
% to show all the authors. Please do not make the titlebox
% smaller than 5cm (the original size); we will check this
% in the camera-ready version and ask you to change it back.

\newcommand\BibTeX{B{\sc ib}\TeX}

\title{Vocabby: Vocabulary learning with in the context of user domain}

\author{Haemanth Santhi Ponnusamy \\
  Department of computational linguistics, \\
  Eberhard Karls University of Tuebingen,\\
  {\tt email@domain} \\\And
  Himanshu Bansal \\
  Department of computational linguistics, \\
  Eberhard Karls University of Tuebingen,\\
  {\tt email@domain} \\}

\date{}

\begin{document}
\maketitle
\begin{abstract}
  This is an vocabulary learning application for a targeted set of user
  who already posses some basic skills of the English language and
  bored of the traditional method of learning new word in the language. This
  application provides the ability to upload text of user preference
  such as a story book, research writing etc... All activities generated for
  vocabulary learning and their distractor are closed to the chosen content.
\end{abstract}

\section{Introduction}
The vocabulary learning is a never ending task. Since the languages are vast
and continuously evolving. Majority of the tools for vocabulary learning takes
the user through a pre defined set of word groups which is constructed into a
hierarchy of word in the language.(motivation) The main drawback of this approach are, it
takes too long for someone to get a feel of progress, it might not be the
order in which the user wants to learn words and the words might not be
introduced in the context that the user is interested in. All the above reasons
makes the user feel boring and finally quit the habit of learning in few days
or weeks. (content creation) In order to keeps the user of different kind
motivated, the developer has to manually/automatically generate lot of content
to show the examples to all the words in the language in multiple context.
We attempt to overcome these issues of creating tool for vocabulary learning
by allowing the user to choose content their interest and we use that to
generate the content for the vocabulary learning activities. By this way one
could choose to improve vocabularies used in a specific domain with specific
context. They could move on to a new area once they complete that.
(Indefinite space) So the problem of vocabulary learning scales down from an
indefinite space to a finite space.

\section{Related works}
Details of some of the existing tools goes here.


\section{Our application}

\subsection{Building vocabulary}
The vocabulary is built from the user text.

\subsubsection{Candidates}
All the words in the user text doesn't need to be addressed. For examples 
function words (the, of, in, on etc...), rare/very less frequent words and
improperly parsed words. So the words occurring below the frequency of 10 and
the stop words are eliminated. 

Also in-order to differentiate between the same words occurring in the
different parts of speech. The words are represented as a tuple of word and its
POS tag.
\begin{center}(word, POS tag)\end{center}

\subsubsection{Occurrence map}
All the instance of the candidate words are cleaned and mapped to the words.
So each word is mapped to all the sentences in which it occur and also each
sentence is mapped to all the words it contains.

\subsubsection{Complexity}
There are lot of methods to measure the complexity of a word. We choose frequency.
The frequency of usage of a word in the SUBTELEXUS as a reference

\subsection{Creating structure}

\subsubsection{Family}

\subsubsection{Network}

\subsection{Learner model}
\subsection{Content organization}
\subsubsection{Books}
\subsubsection{Bookshelf}

\subsection{Details on learning}
\subsubsection{Tutor}
\subsubsection{Session}


\subsection{Activity}
\subsection{Update rule}


\section*{Acknowledgments}

% include your own bib file like this:
%\bibliographystyle{acl}
%\bibliography{acl2018}
\bibliography{acl2018}
\bibliographystyle{acl_natbib}

\end{document}
