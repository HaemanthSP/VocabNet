%
% File acl2018.tex
%
%% Based on the style files for ACL-2017, with some changes, which were, in turn,
%% Based on the style files for ACL-2015, with some improvements
%%  taken from the NAACL-2016 style
%% Based on the style files for ACL-2014, which were, in turn,
%% based on ACL-2013, ACL-2012, ACL-2011, ACL-2010, ACL-IJCNLP-2009,
%% EACL-2009, IJCNLP-2008...
%% Based on the style files for EACL 2006 by 
%%e.agirre@ehu.es or Sergi.Balari@uab.es
%% and that of ACL 08 by Joakim Nivre and Noah Smith

\documentclass[11pt,a4paper]{article}
\usepackage[hyperref]{vocabbysty}
\usepackage{times}
\usepackage{latexsym}

\usepackage{url}

\aclfinalcopy % Uncomment this line for the final submission
\def\aclpaperid{***} %  Enter the acl Paper ID here

%\setlength\titlebox{5cm}
% You can expand the titlebox if you need extra space
% to show all the authors. Please do not make the titlebox
% smaller than 5cm (the original size); we will check this
% in the camera-ready version and ask you to change it back.

\newcommand\BibTeX{B{\sc ib}\TeX}

\title{Vocabby: Vocabulary learning with in the context of user domain}

\author{Haemanth Santhi Ponnusamy \\
  {\tt hameanthsp@gmail.com} \\\And
  Himanshu Bansal \\
  {\tt email@domain}}

\date{}

\begin{document}
\maketitle
\begin{abstract}
  We propose a vocabulary learning application targeted on the set of user
  who already posses some basic skills in a language and bored of the
  traditional method of learning words in the language. Our design 
  efficiently structure the vocabulary space and track the user performance.
  Also the design provides a great advantage in auto-generating the content
  for the entire learning process with almost zero human effort. We allow the
  user to provide the text of there own interest. All the activities and
  feedbacks are generated only from the user text.

\end{abstract}

\section{Introduction}
Vocabulary learning is an open ended task as the languages are vast and still
evolving with the addition of new words now and then. This makes it hard for the
user to get a sense of progress. Many of the vocabulary learning application 
starts from very basic words in the language. Which makes more sense for the 
beginners, but not for our target users. They might want to start at an
intermediate level. Also the example sentences used in these tools as a context
of a word are mostly very generic and some time unnatural. This nature is due to the
fact that those application are designed to support users irrespective of their
language competence, background, area of interest, learning goals. Also It is
hard for the developers to manually create content to satisfy all kind of users.

In this paper we propose an approach to overcome all the above mentioned
difficulties to build an application that could potentially satisfy our targeted
users. We do this by partially sharing the problem with the user to choose the 
text of their interest and reading level. Then we process the chosen text to
select the candidate words and build a network of candidates to efficiently model
the vocabulary of the text and generate useful activities from them. This help
the user in defining the definite space to master. By this way one
could choose to learn words used in a specific domain with specific context.

\section{Related works}
Details of some of the existing tools goes here.


\section{Our application}

\subsection{Building vocabulary}
The raw input text from the user is structured into a meaningful form that is
convenient for vocabulary learning. As we use some statistical feature to
process the user text, the quality of structured data is directly
proportional to the size of the user text. The process of building the
vocabulary with respect to the user text is independent of the user. We focus
to capture the entire vocabulary structure of the text. Thus any new user can
selc

\subsubsection{Candidates}
All the words in the language are not equally important. Given that the users
already know some basics of the language, we could eliminate the most frequent, 
functional words (the, of, in, on etc...), rare/very less frequent words and
improperly parsed words. So the words occurring below the frequency of 10 and
the stop words(\url{https://github.com/explosion/spaCy/blob/master/spacy/lang/en/stop_words.py}) are eliminated to handle the above cases. 

The words that occur in different parts of the speech could posses different sense.
In-order to differentiate between them. We represent each candidate word as a
pair of word and its POS tag.
\begin{center}(word, POS tag)\end{center}

\subsubsection{Sentences}
All the sentences in which the candidate words occur are cleaned and mapped to
the corresponding candidate word. So each candidate word is mapped to all the sentences
in which it occur and also each sentence is mapped to all the candidate
words it contains.

\subsubsection{Complexity}
There are lot of methods to measure the complexity of a word. We choose frequency.
In a language, we consider the words that are less frequently used are as
more complex / uncommon and the words that occurs more frequent are as easier / well known.
We use the frequencies obtained from SUBTLEX-US~\citet{brysbaert2009moving}, a database of 50 million
words from various English-US movies and TV series subtitles.
\begin{equation}
  C_w = \frac{1}{\log_{10}(freq_w)}
\end{equation}

Where ${freq_w}$ is the average frequency of word $w$ per million words in the
database.

\subsubsection{Vector representation}
Also each of the candidate is mapped with a semantic vector representation which
is obtained from the pre trained model \emph{en\_web\_core\_lg} of \emph{spaCy}
(\url{https://spacy.io/}), a natural language processing library. The main
drawback is that it cannot address the out of vocabulary(OOV) words. Which
could be rectified by training a custom word vectors over a decently sized user content.

\subsection{Creating structure}

\subsubsection{Family}
We group the words into families, similar to ~\citet{bauer1993word} work on word
families but instead of seven sub-groups we form a single group for all the types.
The main intuition of grouping the words into families is that the user can
extrapolate their knowledge of inflections of a language to understand/predict
all the possible forms of an unseen word. Similarly with this setup our system
can extrapolate the mastery of one the member to the entire family. Which could 
drastically reduce the number of interaction that the system need to estimate
the user's vocabulary.

\subsubsection{Network}
Now the families has to co-exist in the space of language(limited by the
user text) as a network like a society. It is a fully connected network with each
family with a different affinity to another. The affinity is a measure of
contextual similarity between the families. We compute cosine similarity 
between the mean of word vectors of all the members of the family to similar
mean vector representation of another family.

\begin{equation}
  V_{F_i} = \frac{1}{n} \sum_{k}^{n} V_{w_k}
\end{equation}

where ${n}$ is the number of elements of the family ${F_i}$.

\begin{equation}
  S_{ij} = \frac{V_{F_i} * V_{F_j}}{\|V_{F_i}\|  \|V_{F_j}\|}
\end{equation}

where, ${S_{ij}}$ is the cosine similarity between the vector representations of
families ${F_i}$ and ${F_j}$.


By this way we create more structure in the space of vocabulary. Which come
handy in many situations like activity creation, updating mastery of each
vocabulary and analysis. This structure helps in further reducing the search space
by allowing the model to get a better inference about learners level with
relatively very less and effective interactions compared to a method of
tracking each word in the vocabulary individually.

\subsection{Content organization}
\subsubsection{Books}
All the processed data such as vocabulary, families, network, sample sentences
are packed into a book instance. This also tracks the meta information such as
Title, Author, Genre, Year and Publisher. This help any future user to select
the processed book directly. This also could help in comparing the performance
of different users on the same book.

\subsubsection{Bookshelf}
All such processed book are organized in multiple bookshelves specific to each
domain similar to the gutenberg project (\url{http://www.gutenberg.org}).

\subsection{Models}
In this work we maintain multiple models to track and update different aspect
of the application.

\subsubsection{Learner}
The learner instance maintain the personal information of the user and tracks
the list of instance of books the user has choose to improve vocabulary and the
progress in them. This also could maintain overall vocabulary knowledge of
the user and customize the activity type, feedback and book suggestions based
on individual needs.

\subsubsection{Tutor}
For each book the user selects, a tutor instance will be created. The main
activities of the tutor are to design a learning session, evaluate the
performance and track the mastery level of the user w.r.t all the vocabulary
in the book. Then again generate a new session based on the updated mastery
levels in the network.

\textbf{Mastery Score:} The value ranges between 0 and 1. Initially it is
assigned to 0.5 to indicate the uncertainty. Based on the performance of the
user it is either increased or decreased by a factor. This approach implicitly
capture the un-visited nodes in the network.

\textbf{Update rule:} As we have built a network of families capturing the
contextual similarity. We can incorporate this into our update rule to update
the mastery scores. When we get some outcome for an activity involving a member
from the family ${F_i}$.

\begin{equation}
  M_j = M_j * (1 + (\alpha * sign * S_{ij}))
\end{equation}

Where ${M_i}$ is the mastery of the family ${F_i}$. ${\alpha}$ is a tunable parameter
for the magnitude of an update. ${sign \epsilon \{-1, +1\}}$ is the direction 
of the update. It depends on the correctness of user response to the corresponding
activity. And ${S_{ij}}$ is the measure of contextual similarity between the two
families ${F_i}$ and ${F_j}$.

\subsubsection{Session}
An instance of session is created by the tutor. Which decided list of word
families to be practiced. The key functions of this model are to deciding the
interaction type (teaching, testing, feedback), compose an interaction with all
required data and handles the flow and closure of the session.

In order to make the learning more efficient the most critical nodes of the
graph are selected for a session. Here, The criticality of a node is decided
based on the intrinsic(complexity) and extrinsic(degree, quality of connection)
nature of the node.


\subsection{Activity}
Since our motive is to reduce the effort for content creation. We generate the
activities on the fly. In this work we generate two type of activities. 

\textbf{fit to context:} The user is prompted to complete 3-4 incomplete
sentences with one among the given list of word suggestions. The options are
chosen to be contextually tight to improve the quality of the activity and
user learning outcome.

\textbf{scrambled word:} The user is prompted to come up with a word from the
set of characters to complete an incomplete sentence. The words with word length
less than 6 characters are allowed to generate this activity. Since the larger
answer words makes it more ambiguous for the user to solve.

Currently, the activity types are chosen in random for the answer words of length
more than 6 characters. This could also be enhanced to chose based on the user
interactions.

\subsection{Distractor selection}
The distractors plays a important role in deciding the quality of the activities.
We take advantage of the network of families we built based on the contextual
closeness to overcome this problem. Here we rank the neighbors of the
answer family and chose a best set below a threshold to avoid the synonyms. From
the best set of families we choose the members which matches the POS tag of the
answer word to make all the distractors coherent.


\subsection{Implementation}

\section{Future Works}
The system has limitation on operating on the words that does not have a learned
representation on the vector space. Which could be rectified by learning a
custom word vector specific to the domain. This feature could create new use case
like a tool to learn jargon specifics to a new domain.

Teaching the word in a session.
Control over the complexity (tunable complexity)
Tracking words at learner level across books (to have a warm start for new book)


\section*{Acknowledgments}

% include your own bib file like this:
\bibliography{vocabby}
\bibliographystyle{vocabby_natbib}
\end{document}
